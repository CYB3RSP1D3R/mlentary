
    \samsection{linear algebra and approximation}
      \samquote{
        Stand firm in your refusal to remain conscious during algebra.
        In real life, I assure you, there is no such thing as algebra.
      }{fran lebowitz}

        Linear algebra is the part of geometry that focuses on when a point is
        the origin, when a `line' is a straight, and when two straight lines
        are parallel.
        %
        Linear algebra thus helps us deal with the preceding pictures
        mathematically and automatically.  The concept of `straight lines'
        gives a simple, flexible model for extrapolation from known points to
        unknown points.  That is intuitively why linear algebra will be crucial
        at every stage of 6.86x.

      \sampassage{column vectors and row vectors} % vectors and co-vectors
        The elements of linear algebra are \textbf{column vectors} and
        \textbf{row vectors}.\bcirc \marginnote{%
          \blarr Though we represent the two similarly in a computer's memory, they
          have different geometric meanings.  We save much anguish by
          remembering the difference.
        }
        We have a set $V$ of ``column vectors''.  We're allowed to find $V$'s
        zero vector and to add or scale vectors in $V$ to get other vectors in
        $V$.  $V$ is the primary object we hold in our mind; perhaps, if we're
        doing image classification, then each column vector represents a
        photograph.  We use the word ``space'' or ``vector space'' when talking
        about $V$ to emphasize that we'd like to exploit visual intuitions when
        analyzing $V$.  In short: we imagine each column vector as a point in
        space, or, if we like, as an arrow from the zero vector to that point.

        Now, associated to $V$ is the set of ``row vectors''.  Under the hood, a
        row vector is a linear function from $V$ to the real numbers $\Rr$.  We
        imagine each row vector not as an arrow but as a ``linear'' heat map or
        altitude map that assigns to each point in $V$ a numeric ``intensity''.
        We can visualize a row vector the same way makers of geographic maps
        do: using contours for where in $V$ the row vector attains values
        $\cdots,-2,-1,0,+1,+2,\cdots$.
        These will be a bunch of uniformly spaced parallel ``planes''.  The
        spacing and orientation of the planes depends on and determines the row
        vector.  In short, we imagine each row vector as a collection of
        parallel planes in space.

        Informally: a column vector is a noun or thing whereas a row vector is
        a adjective or property.  The degree to which a property holds on a
        thing (or a description is true of a thing) is gotten by evaluating the
        row vector on the column vector --- remember, a row vector is a
        function, so we can do this evaluation.  Geometrically, if a row vector
        is a bunch of planes and a column vector is an arrow, the two evaluate
        to a number: the number of planes that the arrow pierces.  Intuitively,
        an example of a column vector might be ``this particular photo of my
        pet cow''; an example of a row vector might be ``redness of the left
        half (of the input photo)''.  If we evaluate this row vector on this
        column vector, then we get a number indicating how intensely true it is
        that the left half of that particular photo is red.\marginnote{%
            We can draw an analogy with syntax vs semantics.  This pair of
            concepts pops up in linguistics, philosophy, circuit engineering,
            quantum physics, and more, but all we need to know is that:
            semantics is about things while syntax is about descriptions of
            things.  The two concepts relate in that, given a set of things, we
            can ask for the set of all descriptions that hold true for all
            those things simultaneously.  And if we have a set of descriptions,
            we can ask for the set of all things that satisfy all those
            descriptions simultaneously.  These two concepts stand in formal
            opposition in the sense that: if we have a set of things and make
            it bigger, then the set of descriptions that apply becomes smaller.
            And vice versa.  Then a column vector is an object of semantics.
            And a row vector is an object of syntax.
        }

      \sampassage{inner products}
        Now, here is the key point: the engine behind generalization in machine
        learning (at least, the machine learning we'll do in Units 1 and 2; and
        less visibly but still truly in more than half of each of Units 3,4,5)
        is the ability to translate between things and properties.  If ``my pet
        cow'' is a thing, then ``similar to my pet cow'' is a property.  The whole
        project of machine learning is to define and implement this word
        ``similar to''.  When we define and implement well, our programs can
        generalize successfully from training examples to new, previously
        unseen situations, since they will be able to see which of the training
        examples the new situations are similar to.  Since ``similar to''
        transforms things to properties, the linear algebra math behind
        ``similar to'' is a function from column vectors to row vectors.  This
        brings us to...

        ... inner products, aka kernels.  An inner product is just a fancy word
        for a (linear) function from column vectors to row vectors.  We
        actually demand that this linear function has two nice properties:
        FIRST, it should have an inverse.  That is, it ought to be a two way
        bridge between column vectors and row vectors, allowing us to translate
        things to descriptions and vice versa.  SECOND, it should be symmetric.
        This means that if we have two things, say ``my pet cow'' and ``my pet
        raccoon'', then the degree to which ``my pet raccoon'' has the property
        ``is similar to my pet cow'' ought to match the degree to which ``my pet
        cow'' has the property ``is similar to my pet raccoon''.  Any invertible,
        symmetric, linear function from column vectors to row vectors is called
        an inner product.  Kernel is a synonym here.\marginnote{%
          Beware: the same word, ``kernel'', has different meanings depending
          on context.
        }

        There are generally infinitely many inner products.  Which one we
        choose changes the generalization properties of our machine learning
        program.  Practically, if we are doing machine learning in a concrete
        situation, then we want to choose an inner product that reflects our
        human intuition and experience and domain knowledge about the right
        notion of ``similarity'' in that situation.

        Any inner product $P$ from column vectors to row vectors induces notions
        of length and angle.  We just define a column vector v's length by
        $\sqrt{P(v)(v)}$.  Call that quantity $\|v\|$.  And we define the angle
        $\alpha(v,w)$ between two non-zero column vectors v,w by
        $P(v)(w)=\|v\|\cdot\|w\|\cdot\cos\alpha(v,w)$.  We make these
        definitions so as to match the Pythagorean theorem from plane geometry.\marginnote{%
          You can google up proofs of the Pythagorean theorem (many are quick and
          beautiful) if you wish to dig deeper.
        }
        So once we choose which inner product we'll use (out of the infinitely
        many choices), we get the concepts of euclidean geometry for free.
        Immediately following from that definition of angle, we get that if two
        column vectors have vanishing inner product (i.e., if $P(v)(w)=0$), then
        those vectors are at right angles (i.e. $\alpha(v)(w)=\pi/2$).

        Now, sometimes (but most of the time not), we are blessed in
        that we know more about our situation than just the space V of things.
        Specifically, V might come with a canonical basis.  This just means
        that V comes marked with "the right" axes with respect to which we
        ought to analyze vectors in V.  In this fortunate case, there is also a
        canonical inner product.  It's called dot product.  Again, I want to
        emphasize that a plain vector space doesn't come with a dot product.
        We need a basis for that.

        The dot product is defined as follows.  Say there are $D$ axes and that
        the "unit" vectors along each axis (aka the basis vector) are named
        $(v_i:0\leq i<D)$.  Then we define $P(v_j)(v_i) = (1 \text{ if } i=j
        \text{ else } 0)$ --- the $1$ expresses that each "unit" basis vector
        ought to have length $1$.  The $0$ expresses that different basis
        vectors ought to be at right angles to each other.  This determines $P$
        on all inputs, by linearity: $P(\sum c^\pr_j v_j)(\sum c_i v_i)=\sum_i
        c^\pr_i c_i$ where $c^\pr_j,c_i$ are numbers.  In short: given a basis,
        there is a unique inner product such that those basis elements all have
        length $1$ and are at right angles to each other.  We call that inner
        product the dot product.

        \attnsam{FILL IN LINEAR DECISION BOUNDARY! (remark on featurization and
        argmax nonlinearities)}

        We may \textbf{evaluate} a row vector on a column vector.  \attnsam{FILL
        IN} A \textbf{dot product} is a way of translating between row and
        column vectors.  \attnsam{FILL IN: DISCUSS GENERALIZATION; (DISCUSS ANGLE, TOO)}

      \sampassage{linear maps}

      \sampassage{singular value decomposition}


