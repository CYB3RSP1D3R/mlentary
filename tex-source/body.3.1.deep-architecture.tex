\objectives{%
  \item
  \item
  \item
}

%-------  _  ------------------------------------------------------------------
\sampassage{lamarckism, locality, dependence}%
The key theme of deep learning is what I'll cheekily call the Lamarckian motto:
\textbf{To build a tool, use it!}.  It's as if hitting a flabby, asymmetrical
hammer against a lot of nails gradually calloused the hammer to be exactly the
right hardness and shape to drive the nails straight and deep, or as if we
could lift ourselves by our bootstraps.  For example, in last section's shallow
neural network we wanted to build a featurization useful for linear
classification.  To \emph{build} that featurization, we \emph{used} it: we
piped a differentiably parameterized featurizing layer's output into a linear
classifier and we trained both the featurizer and the classifier using gradient
descent.

We can summarize the situation by drawing learned functions as dashed arrows
and hardcoded functions as solid arrows (see margin).
%\bovinenote{%
%  (this notation, uncommon in ML research, is useful showing relationships at a
%  glance).
%}
Thus, we have a natural distribution over \emph{labeled images}, from which
one function maps to labels and another maps to images (two solid arrows).  Our
shallow neural network consists of a featurizer and a classifier (two dashed
arrows).  We can get from a labeled image to a label by two paths: (a) directly,
or (b) by mapping to just the image, then featurizing, then classifying.  The
condition we use to frame our machine learning problem ---
and the objective of gradient descent --- is that the function (a) and the
composition-of-functions (b) should be nearly equal when evaluated on naturally
occuring labeled images.
\begin{marginfigure}[-3.5cm]
$$
{\normalsize
\begin{tikzcd}
\text{labeled image} \arrow[d] \arrow[rr] &  & \text{label}                    \\
\text{image} \arrow[rr, dashed]           &  & \text{feat.s} \arrow[u, dashed]
\end{tikzcd}
}
$$
\caption{Architecture diagram of shallow neural classifier.  There are two ways to get from the upper left to the lower right corner; we want them to agree.}
\end{marginfigure}
That's all a complicated way of saying that we can improve raw features into
classifier-friendlier features by gradient descent.  Gradient descent on
what?  Not on the raw-to-friendly function in isolation, but on its composition
with a classifier.  We can evaluate how well the composition fits the training
data but not how well the raw-to-friendly function by itself fits the data.\bovinenote{%
  There aren't any paths in the diagram that, like the diagram's bottom edge,
  go from images to features!  So there's no path to directly compare that
  function to.  This talk about paths is a complicated way of saying something
  that here is obvious.  But path language can clarify more complicated situations.
}
This is one of multiple connotations
of the jargon adjective \textbf{end-to-end} when applied to architecture.

So we can improve raw features into classifier-friendlier features.  If
something is fun, let's try it more!  Let's do multiple layers of improvement.
That is, instead of learning to turn raw inputs into features that are easy
to classify, let's learn to turn raw inputs into ``low-level features'' that
are easy to turn into ``high-level features'' that in turn are easy to classify.
We can keep going.  Here is how the idea looks if we have low, medium, and
high features (as before, dashed arrows
depict learned functions):
$$
\begin{tikzcd}
\text{labeled image} \arrow[rrr] \arrow[d] &                                  &                                     & \text{label}                      \\
\text{image} \arrow[r, dashed]             & \text{low f.s} \arrow[r, dashed] & \text{medium f.s} \arrow[r, dashed] & \text{high f.s} \arrow[u, dashed]
\end{tikzcd}
$$

Concretely, this binary classifier could have a formula such
as:\bovinenote{%
  Here, $\sigma$ is the logistic function, $f$ is an nonlinearity such as
  componentwise-leaky-ReLU-followed-by-bias-trick, and $A,B,C,D$ are matrices.
  We write $\cdot$ instead of $\cdot$ for function composition, to reduce
  clutter.  We annotate functions and matrices by a superscript
  $\text{dimension of output}$ and a subscript $\text{dimension of input}$.
}
$$
  \hat p(y\!=\!+1\,|\,x) \,=\,
  (\sigma^{1}_{1} \cdot
  A^{1}_{h+1} \cdot
  f^{h+1}_{h} \cdot
  B^{h}_{\tilde h+1} \cdot
  f^{\tilde h+1}_{\tilde h} \cdot
  C^{\tilde h}_{\hat h+1} \cdot
  f^{\hat h+1}_{\hat h} \cdot
  D^{\hat h}_{d})(x)
$$
Here, $f(D(x))$ gives the low-level features, $f(C(f(D(x))))$ gives the
medium-level features, and so on.  The matrix $A$ does familiar Unit-1-style
linear classification of the super-duper learned features $f(B(f(C(f(D(x))))))$.
%
Intuitively, a single featurization layer such as $f\cdot D$ can't do much by
itself (at least, for computationally practical hidden dimensions and training
cycles), so sequencing such layers helps simple transforms to accumulate into
complex transforms.  It's a bit like origami --- simple folds build on simple
folds to give a beautiful crane.  By increasing depth, we give gradient descent
a deeper hierarchy of folds to play around with as it tries to invent a crane.

\attnsam{I'm unhappy with the lack of categorical product below for the collection
of features as inputs to `high`}
%$$
%\begin{tikzcd}
%\text{img} \arrow[r] \arrow[rd] \arrow[rdd] \arrow[rrrr, "training data" description, bend left] & \text{northeast} \arrow[r, dashed] & \text{NE low} \arrow[r, dashed] & \text{NE medium} \arrow[rd] & \text{label}                  \\
%                                                                                                 & \text{northwest} \arrow[r, dashed] & \text{NW low} \arrow[r, dashed] & \text{NW medium} \arrow[r]  & \text{high} \arrow[u, dashed] \\
%                                                                                                 & \text{southwest} \arrow[r, dashed] & \text{SW low} \arrow[r, dashed] & \text{SW medium} \arrow[ru] &
%\end{tikzcd}
%$$

\attnsam{in subsequent diagrams, single arrows may have many layers}

On with examples of the Lamarckian motto of building parts by exercising them
in context.  Let's say we want to learn how to turn an image of a natural scene
(say, a photo of some furniture) into a depth map\bovinenote{i.e., an array
that says for each pixel in the original photo how many meters away from the
camera the surface depicted by that pixel is; for convenience we'll also regard
the depth map as containing the data of the original image, too}.  As is often
the case in ML, we have huge amounts of data \emph{but not huge amounts of data
directly and reliably labeled for our task of interest}.  Perhaps instead we
have just plain images, or maybe, slightly better, pairs of slightly different
shots of the same scenes.  We want to be clever about squeezing learning signals
from this data.  From an individual image we might not always be able
to specify the depth map uniquely, but at least we should have a consistency
condition: there ought to exist some 3d mesh shape that induces the depth map and
that renders to the given image (this rules out depth maps that are television-noise or M.C.Escherian,
but it imposes constraints on our learned functions beyond this).  So
let's \emph{learn} a depth-to-3dshape function so that we may express this consistency
condition.  It's okay if the learned function ends up being imperfect; it's
a helper function to refine the learning of our depth map.
%
And
a pair of slightly different shots of the same scene should in principle usually
have enough information to determine the depth map for both shots, as well as
determining a small rotation to turn one shot into the other shot.  So we have
two kinds of consistency condition:
% Synthetic data!!
$$
\begin{tikzcd}
\text{img} \arrow[rrr, dashed]                                &                                                                                 &                                                   & \text{depth} \arrow[lld, dashed]                                                            \\
\text{pair of views} \arrow[d] \arrow[u] \arrow[rrrd, dashed] & \text{3d shape} \arrow[r, equal] \arrow[lu, "\texttt{render}", sloped] & \text{3d shape} \arrow[ru, "\texttt{project}", sloped] &                                                                                             \\
\text{img'}                                                   &                                                                                 &                                                   & {\text{img, depth, rotn}} \arrow[uu] \arrow[lll, "\texttt{parallax}", sloped] \arrow[lu, dashed]
\end{tikzcd}
$$

Zooming out, this key theme of deep learning suggests that we can
build sophisticated systems by imposing consistency conditions between their
parts.

\newpage
Here are six further examples of the key theme:
\par\noindent
\begin{table}
\centering
\begin{tabular}{cc}
%\textsc{}
%&
%\textsc{text summarization}
%\\
%\\
\textsc{unpaired translation}
&
\textsc{robust classification}
\\
$$
\begin{tikzcd}
\text{english} \arrow[rr, dashed] \arrow[dd, equal]       &                                                        & \text{french} \arrow[dd, equal]                    \\
                                                          & \{\star\} \arrow[lu] \arrow[rd] &                                                          \\
\text{english}                                      &                                                        & \text{french} \arrow[ll, dashed]
\end{tikzcd}
$$
    &
$$
    \begin{tikzcd}[column sep=1.25em]
    \text{labeled img} \arrow[rrr, equal] \arrow[ddd] &  &  & \text{labeled img} \arrow[ddd, "\texttt{parse}", dashed,sloped] \\
                                                                                                                                    &  &  &                                                                                \\
                                                                                                                                    &  &  &                                                                                \\
\text{image} \arrow[rrruuu, dashed]                                                        &  &  & \text{orient'n} \arrow[lll, "\texttt{render}",sloped]
\end{tikzcd}
$$
\\
\\
\textsc{word embedding}
&
\textsc{concept hints}
\\
$$
\begin{tikzcd}[row sep=3em]
\text{sentence} \arrow[d, "\texttt{redact}", sloped] \arrow[rrr] &  &  & \text{sentence}                                                                         \\
\text{word list} \arrow[rrr, "\texttt{embed}", dashed]               &  &  & \text{featurized} \arrow[u, dashed] \arrow[d] \\
\text{analogy Q} \arrow[u] \arrow[rrr]                                        &  &  & \text{analogy A}
\end{tikzcd}
$$
&
$$
\begin{tikzcd}[column sep=0, row sep=3em]
\text{song audio} \arrow[rd] \arrow[dd] \arrow[rr, dashed] &               & \text{med features} \arrow[dd, dashed] \arrow[ld, dashed] \\
                                                           & \text{chords} &                                                           \\
\text{emotion}                                             &               & \text{hi features} \arrow[ll, dashed] \arrow[lu, dashed]
\end{tikzcd}
$$
\\
\\
\textsc{image generation}
&
\textsc{board gameplay}
\\
$$
\begin{tikzcd}[row sep=3em]
\{\star\} \arrow[d] \arrow[r, equal]                & \{\star\} \arrow[r, equal] \arrow[d]             & \{\star\} \arrow[d]                           \\
\text{small img} \arrow[r, dashed] \arrow[d, equal] & \text{med img} \arrow[r, dashed] \arrow[d, equal] & \text{big img} \arrow[d, equal] \\
\text{small  img}                                                 & \text{med img} \arrow[l]                                        & \text{big img} \arrow[l]
\end{tikzcd}
$$
    &
    $$
    \begin{tikzcd}[column sep=1.25em, row sep=3em]
\text{board} \arrow[rr, "\texttt{move}"] \arrow[rd] \arrow[dd, dashed] &              & \text{boards} \arrow[dd, dashed]                \\
                                                              & \text{wins?} &                                                 \\
\text{value} \arrow[ru]                                       &              & \text{values} \arrow[ll, "\texttt{max or min}"]
\end{tikzcd}
    $$
\end{tabular}
\end{table}

%Dall-E
%GPT

\newpage
%-------  _  ------------------------------------------------------------------
\sampassage{symmetry, convolution, attention}

%-------  _  ------------------------------------------------------------------
\sampassage{hierarchy, representations, transfer}

%-------  _  ------------------------------------------------------------------
\sampassage{uncertainty, structure, generation}


% =============================================================================
% ==  _  ======================================================================
% =============================================================================


\objectives{%
  \item 
  \item 
  \item 
}

%-------  _  ------------------------------------------------------------------
\sampassage{backpropagation and gradient descent}

%-------  _  ------------------------------------------------------------------
\sampassage{visualizing learning dynamics}

%-------  _  ------------------------------------------------------------------
\sampassage{visualizing learned represenations}

%-------  _  ------------------------------------------------------------------
\sampassage{example code}

