%~~~~~~~~~~~~~~~~~~~~~~~~~~~~~~~~~~~~~~~~~~~~~~~~~~~~~~~~~~~~~~~~~~~~~~~~~~~~~~
%~~~~~~~~~~~~~  1.9. workflow  ~~~~~~~~~~~~~~~~~~~~~~~~~~~~~~~~~~~~~~~~~~~~~~~~

      \sampassage{workflow}
      %\sampassage{workflow: framing}
        We first \emph{frame}: what data will help us solve what problem?  To
        do this, we \emph{factor} our complex prediction problem into simple
        classification or regression problems; randomly \emph{split} the
        resulting example pairs into training, dev(elopment), and testing sets;
        and \emph{visualize} the training data to weigh our intuitions.

      %\sampassage{workflow: modeling}
        Next, we \emph{model}: we present the data to the computer so that
        true patterns are more easily found.
        %
        Here we inject our \emph{domain knowledge} --- our human experience and
        intuition about which factors are likely to help with prediction.
        %
        Modeling includes \emph{featurizing} our inputs and choosing
        appropriate \emph{priors} and \emph{symmetries}.

      %\sampassage{workflow: training}
        During \emph{training}, the computer searches among candidate patterns
        for one that explains the examples relatively well.
        We used brute force above; we'll soon learn faster algorithms
        such as \emph{gradient descent} on the training set for parameter
        selection and \emph{random grid search} on the dev set for
        hyperparameter selection.

      %\sampassage{workflow: harvesting}
        Finally, we may \emph{harvest}: we derive insights from the pattern
        itself\bcirc\marginnote{%
            \blarr which factors ended up being most important?
        }
        and we predict outputs for to fresh inputs.
        %
        Qualifying both applications is the pattern's quality.  To assess this,
        we measure its accuracy on our held-out testing data.


