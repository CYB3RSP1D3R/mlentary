\documentclass[12pt]{article}

% author:   samtenka
% change:   2023-03-28
% create:   2022-05-11

\newcommand{\picturedw}[2]{\includegraphics[width=#1]{#2}}
\newcommand{\picturew}[2]{\includegraphics[width=#1]{figures/#2}}
\newcommand{\pictureh}[2]{\includegraphics[height=#1]{figures/#2}}

%==============================================================================
%====  0.  DOCUMENT SETTINGS  ================================================
%==============================================================================

%~~~~~~~~~~~~~~~~~~~~~~~~~~~~~~~~~~~~~~~~~~~~~~~~~~~~~~~~~~~~~~~~~~~~~~~~~~~~~~
%~~~~~~~~~~~~~  0.0. About this Exposition  ~~~~~~~~~~~~~~~~~~~~~~~~~~~~~~~~~~~

%---------------------  0.0.1. math packages  ---------------------------------
\newcommand\hmmax{0} % to allow for more fonts
\newcommand\bmmax{0} % to allow for more fonts
\usepackage{amsmath, amssymb, amsthm, mathtools}
\usepackage{bm}
\usepackage{euler}

\usepackage{array}   % for \newcolumntype macro
\newcolumntype{L}{>{$}l<{$}} % math-mode version of "l" column type
\newcolumntype{C}{>{$}c<{$}} % math-mode version of "c" column type
\newcolumntype{R}{>{$}r<{$}} % math-mode version of "r" column type

\newcolumntype{S}{ >{\centering\arraybackslash} m{3cm} } % vertically and horizontally centered

%---------------------  0.0.2. graphics packages  -----------------------------
\usepackage{graphicx, xcolor}
\usepackage{float, capt-of}
\usepackage{soul}

%---------------------  0.0.3. packages for fancy text  -----------------------
\usepackage{enumitem}\setlist{nosep}
\usepackage{listings}
\usepackage{xstring}
\usepackage{fontawesome5}

%---------------------  0.043. colors  ----------------------------------------

% NOTE: we want to cater to colorblind readers

% (LIGHT, MEDIUM, DARK) x (BLUE, ORANGE)
\definecolor{msky}{rgb}{0.62, 0.82, 0.94} \newcommand{\sky}{\color{msky}}
\definecolor{mpch}{rgb}{0.98, 0.86, 0.62} \newcommand{\pch}{\color{mpch}}

\definecolor{mblu}{rgb}{0.05, 0.55, 0.85} \newcommand{\blu}{\color{mblu}}
\definecolor{mrng}{rgb}{0.95, 0.65, 0.05} \newcommand{\rng}{\color{mrng}}

\definecolor{msea}{rgb}{0.02, 0.22, 0.34} \newcommand{\sea}{\color{msea}}
\definecolor{mbro}{rgb}{0.38, 0.26, 0.02} \newcommand{\bro}{\color{mbro}}

% SHADES:
\definecolor{mgre}{rgb}{0.55, 0.55, 0.55} \newcommand{\gre}{\color{mgre}}
\definecolor{mdgre}{rgb}{0.35, 0.35, 0.35} \newcommand{\dgre}{\color{mdgre}}

% UNFRIENDLY:
\definecolor{mred}{rgb}{1.00, 0.00, 0.00} \newcommand{\red}{\color{mred}}

%~~~~~~~~~~~~~~~~~~~~~~~~~~~~~~~~~~~~~~~~~~~~~~~~~~~~~~~~~~~~~~~~~~~~~~~~~~~~~~
%~~~~~~~~~~~~~  0.1. Headers and References  ~~~~~~~~~~~~~~~~~~~~~~~~~~~~~~~~~~

%---------------------  0.1.0. intra-document references  ---------------------
\newcommand{\offour}[1]{
    {\tiny \raisebox{0.04cm}{\scalebox{0.9}{$\substack{
        \IfSubStr{#1}{0}{{\blacksquare}}{\square}
        \IfSubStr{#1}{1}{{\blacksquare}}{\square} \\ 
        \IfSubStr{#1}{2}{{\blacksquare}}{\square}
        \IfSubStr{#1}{3}{{\blacksquare}}{\square}
    }$}}}%
}

\newcommand{\offourline}[1]{
    {\tiny \raisebox{0.04cm}{\scalebox{0.9}{$\substack{
        \IfSubStr{#1}{0}{{\blacksquare}}{\square}
        \IfSubStr{#1}{1}{{\blacksquare}}{\square}
        \IfSubStr{#1}{2}{{\blacksquare}}{\square}
        \IfSubStr{#1}{3}{{\blacksquare}}{\square}
    }$}}}%
}
\newcommand{\notesam}[1]{{\blu \textsf{#1}}}
\newcommand{\attn}[1]{{\blu \textsf{#1}}}
\newcommand{\attnsam}[1]{{\red \textsf{#1}}}
%\newcommand{\attnsam}[1]{}%{\red \textsf{#1}}}

\newcommand{\blarr}{\hspace{-0.15cm}${\blu \leftarrow}\,$}
\newcommand{\bcirc}{${\blu ^\circ}$}
\newcommand{\bovinenote}[1]{\bcirc\marginnote{\blarr #1}}

%---------------------  0.1.1. table of contents helpers  ---------------------
\newcommand{\phdot}{\phantom{.}}

%---------------------  0.1.2. section headers  -------------------------------
\newcommand{\samtitle} [1]{
  \par\noindent{\Huge \sf \blu #1}
  \vspace{0.4cm}
}

\newcommand{\samquote} [2]{
    \marginnote[-0.4cm]{\begin{flushright}
    %\scriptsize
        \gre {\it #1} \\ --- #2
    \end{flushright}}
}

\newcommand{\sampart} [1]{
  \vspace{0.5cm}
  \par\noindent{\LARGE \sf \blu #1}
  \vspace{0.1cm}\par
}

\newcommand{\samsection}[1]{
  \vspace{0.3cm}
  \par\noindent{\Large \sf \blu #1}
  \vspace{0.1cm}\par
}

\newcommand{\sampassage}[1]{
   \vspace{0.1cm}
   \par\noindent{\hspace{-2cm}\normalsize \sc \gre #1} ---
}

%---------------------  0.1.3. clear the bibliography's header  ---------------
\usepackage{etoolbox}
\patchcmd{\thebibliography}{\section*{\refname}}{}{}{}

%~~~~~~~~~~~~~~~~~~~~~~~~~~~~~~~~~~~~~~~~~~~~~~~~~~~~~~~~~~~~~~~~~~~~~~~~~~~~~~
%~~~~~~~~~~~~~  0.2. Math Symbols and Blocks  ~~~~~~~~~~~~~~~~~~~~~~~~~~~~~~~~~

%---------------------  0.2.0. general math operators  ------------------------
\newcommand{\scirc}{\mathrel{\mathsmaller{\mathsmaller{\mathsmaller{\circ}}}}}
\newcommand{\cmop}[2]{{(#1\!\to\!#2)}}
\newcommand{\pr}{^\prime}
\newcommand{\prpr}{^{\prime\prime}}

\newcommand{\wrap}[1]{\left(#1\right)}

%---------------------  0.2.1. probability symbols  ---------------------------
\newcommand{\KL}{\text{KL}}
\newcommand{\EN}{\text{H}}
\newcommand{\note}[1]{{\blu \textsf{#1}}}

%---------------------  0.2.2. losses averaged in various ways  ---------------
\newcommand{\Ein}  {\text{trn}_{\sS}}
\newcommand{\Einb} {\text{trn}_{\check\sS}}
\newcommand{\Einc} {\text{trn}_{\sS\sqcup \check\sS}}
\newcommand{\Egap} {\text{gap}_{\sS}}
\newcommand{\Eout} {\text{tst}}

%---------------------  0.2.3. double-struck and caligraphic upper letters  ---
\newcommand{\Aa}{\mathbb{A}}\newcommand{\aA}{\mathcal{A}}
\newcommand{\Bb}{\mathbb{B}}\newcommand{\bB}{\mathcal{B}}
\newcommand{\Cc}{\mathbb{C}}\newcommand{\cC}{\mathcal{C}}
\newcommand{\Dd}{\mathbb{D}}\newcommand{\dD}{\mathcal{D}}
\newcommand{\Ee}{\mathbb{E}}\newcommand{\eE}{\mathcal{E}}
\newcommand{\Ff}{\mathbb{F}}\newcommand{\fF}{\mathcal{F}}
\newcommand{\Gg}{\mathbb{G}}\newcommand{\gG}{\mathcal{G}}
\newcommand{\Hh}{\mathbb{H}}\newcommand{\hH}{\mathcal{H}}
\newcommand{\Ii}{\mathbb{I}}\newcommand{\iI}{\mathcal{I}}
\newcommand{\Jj}{\mathbb{J}}\newcommand{\jJ}{\mathcal{J}}
\newcommand{\Kk}{\mathbb{K}}\newcommand{\kK}{\mathcal{K}}
\newcommand{\Ll}{\mathbb{L}}\newcommand{\lL}{\mathcal{L}}
\newcommand{\Mm}{\mathbb{M}}\newcommand{\mM}{\mathcal{M}}
\newcommand{\Nn}{\mathbb{N}}\newcommand{\nN}{\mathcal{N}}
\newcommand{\Oo}{\mathbb{O}}\newcommand{\oO}{\mathcal{O}}
\newcommand{\Pp}{\mathbb{P}}\newcommand{\pP}{\mathcal{P}}
\newcommand{\Qq}{\mathbb{Q}}\newcommand{\qQ}{\mathcal{Q}}
\newcommand{\Rr}{\mathbb{R}}\newcommand{\rR}{\mathcal{R}}
\newcommand{\Ss}{\mathbb{S}}\newcommand{\sS}{\mathcal{S}}
\newcommand{\Tt}{\mathbb{T}}\newcommand{\tT}{\mathcal{T}}
\newcommand{\Uu}{\mathbb{U}}\newcommand{\uU}{\mathcal{U}}
\newcommand{\Vv}{\mathbb{V}}\newcommand{\vV}{\mathcal{V}}
\newcommand{\Ww}{\mathbb{W}}\newcommand{\wW}{\mathcal{W}}
\newcommand{\Xx}{\mathbb{X}}\newcommand{\xX}{\mathcal{X}}
\newcommand{\Yy}{\mathbb{Y}}\newcommand{\yY}{\mathcal{Y}}
\newcommand{\Zz}{\mathbb{Z}}\newcommand{\zZ}{\mathcal{Z}}

%---------------------  0.2.4. sans serif and frak lower letters  -------------
\newcommand{\sfa}{\mathsf{a}}\newcommand{\fra}{\mathfrak{a}}
\newcommand{\sfb}{\mathsf{b}}\newcommand{\frb}{\mathfrak{b}}
\newcommand{\sfc}{\mathsf{c}}\newcommand{\frc}{\mathfrak{c}}
\newcommand{\sfd}{\mathsf{d}}\newcommand{\frd}{\mathfrak{d}}
\newcommand{\sfe}{\mathsf{e}}\newcommand{\fre}{\mathfrak{e}}
\newcommand{\sff}{\mathsf{f}}\newcommand{\frf}{\mathfrak{f}}
\newcommand{\sfg}{\mathsf{g}}\newcommand{\frg}{\mathfrak{g}}
\newcommand{\sfh}{\mathsf{h}}\newcommand{\frh}{\mathfrak{h}}
\newcommand{\sfi}{\mathsf{i}}\newcommand{\fri}{\mathfrak{i}}
\newcommand{\sfj}{\mathsf{j}}\newcommand{\frj}{\mathfrak{j}}
\newcommand{\sfk}{\mathsf{k}}\newcommand{\frk}{\mathfrak{k}}
\newcommand{\sfl}{\mathsf{l}}\newcommand{\frl}{\mathfrak{l}}
\newcommand{\sfm}{\mathsf{m}}\newcommand{\frm}{\mathfrak{m}}
\newcommand{\sfn}{\mathsf{n}}\newcommand{\frn}{\mathfrak{n}}
\newcommand{\sfo}{\mathsf{o}}\newcommand{\fro}{\mathfrak{o}}
\newcommand{\sfp}{\mathsf{p}}\newcommand{\frp}{\mathfrak{p}}
\newcommand{\sfq}{\mathsf{q}}\newcommand{\frq}{\mathfrak{q}}
\newcommand{\sfr}{\mathsf{r}}\newcommand{\frr}{\mathfrak{r}}
\newcommand{\sfs}{\mathsf{s}}\newcommand{\frs}{\mathfrak{s}}
\newcommand{\sft}{\mathsf{t}}\newcommand{\frt}{\mathfrak{t}}
\newcommand{\sfu}{\mathsf{u}}\newcommand{\fru}{\mathfrak{u}}
\newcommand{\sfv}{\mathsf{v}}\newcommand{\frv}{\mathfrak{v}}
\newcommand{\sfw}{\mathsf{w}}\newcommand{\frw}{\mathfrak{w}}
\newcommand{\sfx}{\mathsf{x}}\newcommand{\frx}{\mathfrak{x}}
\newcommand{\sfy}{\mathsf{y}}\newcommand{\fry}{\mathfrak{y}}
\newcommand{\sfz}{\mathsf{z}}\newcommand{\frz}{\mathfrak{z}}

%---------------------  0.2.5. math environments  -----------------------------
\newtheorem*{qst}{Question}
\newtheorem*{thm}{Theorem}
\newtheorem*{lem}{Lemma}
% ...
\theoremstyle{definition}
\newtheorem*{dfn}{Definition}

\newcommand{\exercise}[1]{%
  \par\noindent%
  \attn{Food For Thought:} #1%
}
\newcommand{\noparexercise}[1]{%
  \attn{Food For Thought:} #1%
}
\newcommand{\objectives}[1]{%
  \marginnote[-0.2cm]{%
    By the end of this section, you'll be able to
    \begin{itemize}#1\end{itemize}
  }
}




\usepackage{amssymb}
\usepackage{wrapfig}

\begin{document}

\begin{center}
  \huge
    686 course revision, 2023
\end{center}

\newcommand{\defined}[1]{\textbf{#1}}

\part{Motivation}
  \section*{\sc Learner Concerns}


  \section*{\sc Cohesion and Completeness}
    \subsection*{}
    \subsection*{}
    \subsection*{}
    \subsection*{}
    \subsection*{Pointing Students to External Resources.  Clarifying What They Learn on Own Neede}

  \section*{\sc Modern Emphasis}
    \subsection*{Topics to Introduce and/or Emphasize}
      Besides Attention Layers and Variational Autoencoders, the topics to
      introduce and/or emphasize are more in the way of LIGAMENTS to reveal the
      unity of course content and to help learners distilll their
      understanding.  What I want to avoid is a feeling that Units 4
      and 5 are just a grab bag of not-very-related topics.  Helping cohesion
      is a more unified view or emphasis, e.g.\ keeping in mind everything
      as approximate probability maximization etc.

      Overfitting, Regularization, and Priors.
      Role of Featurization
      Probabilistic Models.
      Conditional Bandits (one-state) for comparison to supervised learning.

    \subsection*{Topics to De-emphasize}
      Perceptrons, RNNs (including LSTMS), Kernel trick (especially motivation
      from speed), analytic solution to least squares, K-means and K-medioids,
      Low-rank approximation, collaborative filtering.

      How to De-emphasize?
        E.g. move from lectures to homeworks, mention but don't belabor

    \subsection*{}
    \subsection*{}
    \subsection*{}

  \section*{\sc Non-Busy Work}  %Clear Goals and Clear Point.  No busywork}
    \subsection*{}
    \subsection*{Forum is Hard to Search}
    \subsection*{Revised Projects}
    \subsection*{Less Calculation Homeworks.  Replace Toy Examples by Insightful Toy Examples}
    \subsection*{Projects, Slowness, "Vectorize"}


  \section*{\sc Forum Dynamics}
    \subsection*{}
    \subsection*{}
    \subsection*{}


    \newpage

\part{Course Mechanics}
  \section*{\sc Overview}

    \subsection*{Components}
      The class consists of lectures, plus forum participation, homeworks,
      projects, and a final exam.  We have a \defined{Coding Lecture} in each
      Unit, filling an needed hole.

    \subsection*{Story Arc}

    \subsection*{Learning Objectives}

    \subsection*{Time Commitment}
      This seems about right to me:
      \begin{align*}
           25 \times \text{1 hr } -& \text{watch each of 25 lectures}
        \\ 25 \times \text{2 hrs} -& \text{study each of 25 lectures and summarize and do forum Q/A}
        \\  5 \times \text{8 hrs} -& \text{complete each of 5 homeworks}
        \\  5 \times \text{8 hrs} -& \text{complete each of 5 projects}
        \\  1 \times \text{6 hrs} -& \text{study for the final}
        \\  1 \times \text{2 hrs} -& \text{complete the final}
        \\  1 \times \text{5 hrs} -& \text{overhead (review prereqs, install packages, etc)}
      \end{align*}
      ... so 168 hours total (about 14 hrs per week over 12 weeks).
      %
      But learners with different levels of preparation and with different
      learning styles might need much less or much more time.

    \subsection*{Grading}
      What's graded is forum participation, homeworks, projects, and a final
      exam.

      \subsubsection*{Homeworks vs Projects}
        FILLIN

        The point of projects is to get learners some hands-on experience (a)
        seeing ``the whole car'' built from all its little pieces exposed and
        viewable in context and (b) learning to improve and drive
        the car.  Ways to test a project are:
        (a) asking for program output,
        (b) auto-grader unit tests,
        (c) prediction challenge,
        (d) honor system self-assessment or peer assessment.
        %
        These differ in reliability, the flexibility and creativity they
        permit, how heavy-weight our automated grading infrastructure has to
        be, and how frustrating it can be for the students to fight with the
        grader.

        The real value of the projects is hard to grade.  I mean that I'm
        having a hard time thinking of a scalable grading scheme (besides maybe
        getting student pairs to peer review each other) whose numeric grades
        relate causally with the kind of project-engagement that students
        actually learn from.

        Each Unit will have two homeworks and two projects.

        I want to shift weight (slightly and not completely) away from
        the auto-grader.

        ProjectA (``car build from parts'') involves adding a bit of code to
        pre-existing code.  The main way we test is by asking learners for
        program output (e.g., numbers, rankings, including from helper
        functions) and by auto-grading.

        ProjectB (``improving and driving the car'') involves using and
        fiddling with already-functioning code.  The main way we test is by a
        prediction challenge.

      \subsubsection*{Forum Participation through Lecture Summaries}
        Forum participation centers on an unusual idea called
        \defined{lecture summaries}: after each of the $\sim 20$ lectures, a
        learner must summarize the lecture in their own words (roughly two
        paragraphs, three hundred words), including their confusions.
        This is an excellent way to get learners to really ponder and internalize
        lecture material; it also can spark forum discussion and help us diagnose
        and address any widespread confusions before project time.

        A lecture summary is adequate if upon brief inspection it plausibly
        evidences that the learner worked hard to digest the lecture and to
        clearly record their impression of and confusions from that lecture.
        Provided among the materials at the start of the class will be examples
        of adequate lecture summaries.  During finals week, a TA will randomly
        select one lecture; then, for each learner, the TA will check whether
        that learner's lecture summary for that selected lecture is adequate.
        For 1000 learners and 10 seconds per summary, this is a 3 hour job.
        This process should be helped by a scraper and grader script so that
        the TA can just press Q for inadequate and P for adequate and
        immediately (without waiting for edx loading) move to the next learner.
        So each learner gets either full or zero points for this grade
        component.

        This is a random procedure; a learner's expected forum grade increases
        linearly with the number of lecture summaries they complete.  But it
        ain't a capricious procedure: the variance of a learner's forum grade
        goes to zero the limit of adequate completion of all lecture summaries.

        We also encourage learners to ask and answer thoughtfully and
        frequently on the forum, but this is not graded.

      \subsubsection*{Final Exam}
        There is only one exam: a final exam.  It consists of $\sim 12$
        multiple choice questions, each with about $\sim 5$ choices exactly one
        of which is correct.  A learner gets 1 point for each correct answer
        and 0 points for each incorrect or blank answer.  Very simple.

        This Final Exam is an opportunity to test \defined{integrated
        knowledge} and conceptual (hence transferable) understanding.  So many
        of the questions will involve concepts from multiple units, and, better
        yet, will test the Grand Themes that thread through the whole course.

        Why Multiple Choice?  MC is easier to auto-grade, fits better to the
        conceptual understanding I want to test, and causes less resentment,
        than other exam question formats.  Other formats can cause resentment
        by being confusing as to the exact space of answers to choose from
        (e.g.\ $\texttt{pi}$ vs \texttt{numpy.pi} vs $3.141$ vs $180^\circ$ vs
        $1/2 \text{~turns}$) OR by forcing the learner to cater to unknown
        precision thresholds (significant figures, etc).  The menu of candidate
        answers in an MC problem serves to clarify the question itself.

      \subsubsection*{Formula for Normalized Grades}
        We \defined{normalize} each of $10$ grade components by the average
        score in that component among the top $100$ scores (for a class of
        $1000$ learners, this is the top $10\%$) for that component.  The point
        is to enjoy most of the benefits of curving while avoiding most of the
        downsides.

        Indeed, under this system, a learner at any point in the term has a
        nontrivial lower bound on the grade contribution from the components
        they have completed so far, by ignoring the normalization.  I mean that
        it is reasonable for a learner to pretend there is no normalization at
        all, and that with respect to this the normalization will only help
        them.  Yet, if a component (especially project or exam) is harder than
        we thought, the normalization will save the class's grade.

        I also believe the normalization will not cause perverse competition
        incentives between learners: the way to change the normalization is to
        confuse or to avoid collaborating with top learners, but my model of
        such top learners is that by their nature this injury isn't plausibly
        achievable by others.  This system maintains a strong cooperation
        incentive for the non-top learners: if they can help each other get
        numerically (rather than rank-wise) closer to the top learners, then
        they will improve their grades without harming anyone else's grade.

        In an ideal situation, every learner maintains a not-too-bad baseline
        understanding, so there will be strong clustering (so heavy upward tail
        but those who do less well are very bounded in how badly they do), in
        which case everybody gets an A.
        %
        So instead of the ``evil'' act of forcing there to be at least
        such-and-such many bad grades, this system makes it possible for the
        entire class to earn an A.

        We normalize component-by-component rather than at a coarser level
        in order to maintain proportions between parts.  Otherwise, variance
        in exam grades could dominate everything else etc.

        \begin{align*}
          \text{Grade} =
               ~&10 \cdot {\gre \min(1,~} ( \text{forum score}     ) {\gre / ( \text{average of top $100$ scores for forum}       ))}
            \\+~& 8 \cdot {\gre \min(1,~} ( \text{homework1 score} ) {\gre / ( \text{average of top $100$ scores for homework1}   ))}
            \\+~& 8 \cdot {\gre \min(1,~} ( \text{homework2 score} ) {\gre / ( \text{average of top $100$ scores for homework2}   ))}
            \\+~& 8 \cdot {\gre \min(1,~} ( \text{homework3 score} ) {\gre / ( \text{average of top $100$ scores for homework3}   ))}
            \\+~& 8 \cdot {\gre \min(1,~} ( \text{homework4 score} ) {\gre / ( \text{average of top $100$ scores for homework4}   ))}
            \\+~& 8 \cdot {\gre \min(1,~} ( \text{homework5 score} ) {\gre / ( \text{average of top $100$ scores for homework5}   ))}
            \\+~& 8 \cdot {\gre \min(1,~} ( \text{project1 score}  ) {\gre / ( \text{average of top $100$ scores for project1}    ))}
            \\+~& 8 \cdot {\gre \min(1,~} ( \text{project2 score}  ) {\gre / ( \text{average of top $100$ scores for project2}    ))}
            \\+~& 8 \cdot {\gre \min(1,~} ( \text{project3 score}  ) {\gre / ( \text{average of top $100$ scores for project3}    ))}
            \\+~& 8 \cdot {\gre \min(1,~} ( \text{project4 score}  ) {\gre / ( \text{average of top $100$ scores for project4}    ))}
            \\+~& 8 \cdot {\gre \min(1,~} ( \text{project5 score}  ) {\gre / ( \text{average of top $100$ scores for project5}    ))}
            \\+~&10 \cdot {\gre \min(1,~} ( \text{final score}     ) {\gre / ( \text{average of top $100$ scores for final}       ))}
        \end{align*}
        The maximal Grade is $100$.
        Letter grades are
        \begin{align*}
          A+ & \text{~when Grade is $\geq 95$ else:}\\
          A  & \text{~when Grade is $\geq 80$ else:}\\
          B  & \text{~when Grade is $\geq 65$ else:}\\
          C  & \text{~when Grade is $\geq 50$ else:}\\
          D  & \text{~}
        \end{align*}

    \newpage

\part{Course Content}
  \section*{\sc Unit 0: Prerequisites}
    \subsection*{Coding Lecture 0: Python, Numpy, Pytorch}
      \subsubsection*{Multi-axis arrays.  Speedups.  Index kung-fu.}
        \paragraph{\sf} --- notion of nparray and shape.  to/from list, types
        \paragraph{\sf} --- making arrays: zeros, ones, arange, linspace, eye, diag
        \paragraph{\sf} --- access, slices, setting (mutates cuz is object!)
        \paragraph{\sf} --- shapes to shapes.  flatten, newaxis, transpose
        \paragraph{\sf} --- example: pooling (channel widen) speedup example

      \subsubsection*{Common numpy maps and zips, filters, reduces, and contractions}
        \paragraph{\sf} --- elementwise log, exp; plus, times.  broadcasting (scalar vs vector)
        \paragraph{\sf} --- (in)equality testing (filter), index composition
        \paragraph{\sf} --- mean, variance, stddev, max.  examples along different axes
        \paragraph{\sf} --- dots, matmul, general contraction
        \paragraph{\sf} --- example: one-hot and logistic softmax speedup example (use lambda)

      \subsubsection*{Example: large-numbers dice statistics, plotting}
        \paragraph{\sf} --- generator (randomness and seed!)
        \paragraph{\sf} --- streaming statistics: avg, highwatermark outliers
        \paragraph{\sf} --- plotting
        \paragraph{\sf} --- normalizing (iterated log)
        \paragraph{\sf} --- plotting bells and whistles, saving

      \subsubsection*{Example: (batched) image filter made by hand}
        \paragraph{\sf} --- loading data
        \paragraph{\sf} --- images represented as arrays
        \paragraph{\sf} --- blur filter: average with neighbor; mention boundary choices
        \paragraph{\sf} --- sharpen filter: reverse of blur!
        \paragraph{\sf} --- making everything batchwise

      \subsubsection*{Intro to Pytorch}
        \paragraph{\sf} --- notion of computation graphs
        \paragraph{\sf} --- pytorch tensor vs (and to/from) numpy array)
        \paragraph{\sf} --- exponential decay GD by hand (exercise: do in numpy)
        \paragraph{\sf} --- autodiff
        \paragraph{\sf} --- exponential decay GD by autodiff

    \newpage

    \subsection*{Homework 0a: Math}
      \subsubsection*{Types, Functions and Dependencies, Notation}
      \subsubsection*{Linear Algebra: High Dimensions, Hyperplanes, Linear Maps, Trace and Det; Quadratic Forms, Dot Products, SVD}
      \subsubsection*{Probability: Expectations, Independence, Bayes, Concentration; Coinflips, Gaussians}
      \subsubsection*{Optimization: Visualizing Derivative Rules, Sums of Terms (Constraints); Vectors vs Covectors, Overshooting, Convexity}
      \subsubsection*{Examples: Least Squares; Gaussian Fitting M-Step}
    \newpage

    \subsection*{Homework 0b: Programming}
      \subsubsection*{Matrix Multiply Speed Test}
      \subsubsection*{Numpy Safari / Treasurehunt}
      \subsubsection*{Softmax Speed Test}
      \subsubsection*{Debugging Randomized Code}
      \subsubsection*{Debugging Many-File Codebase}
    \newpage

    %\subsection*{Project 0a: }
    %\subsection*{Project 0b: }

    \newpage

  \section*{\sc Unit 1: Learning from Examples}
    \subsection*{Lecture 1a: What is Machine Learning?}
      \subsubsection*{}
      \subsubsection*{}
      \subsubsection*{}
      \subsubsection*{}
      \subsubsection*{}
    \newpage

    \subsection*{Lecture 1b: A Simple Image Classifier}
      \subsubsection*{}
      \subsubsection*{}
      \subsubsection*{}
      \subsubsection*{}
      \subsubsection*{}
    \newpage

    \subsection*{Lecture 1c: Accelerate Learning using Gradient Descent}
      \subsubsection*{}
      \subsubsection*{}
      \subsubsection*{}
      \subsubsection*{}
      \subsubsection*{}
    \newpage

    \subsection*{Lecture 1d: Training vs Testing Error} % Generalization, Optimization, Approximation
      \subsubsection*{}
      \subsubsection*{}
      \subsubsection*{}
      \subsubsection*{}
      \subsubsection*{}

    \subsection*{Coding Lecture 1: Coding the image classifier from lectures, from scratch.  Examining behavior on a concrete image.}
      \subsubsection*{}
      \subsubsection*{}
      \subsubsection*{}
      \subsubsection*{}
      \subsubsection*{}
    \newpage

    \subsection*{Homework 1a: Hyperplanes and Dataclouds}
      \subsubsection*{}
      \subsubsection*{}
      \subsubsection*{}
      \subsubsection*{}
      \subsubsection*{}
    \newpage

    \subsection*{Homework 1b: Learning.  Weights-vs-Correlations}
      \subsubsection*{}
      \subsubsection*{}
      \subsubsection*{}
      \subsubsection*{}
      \subsubsection*{}
    \newpage

    \subsection*{Project 1a: Text Classification} % not 3-Way.  Just 2-way.
      \subsubsection*{}
      \subsubsection*{}
      \subsubsection*{}
      \subsubsection*{}
      \subsubsection*{}
    \newpage

    \subsection*{Project 1b: Measuring Overfitting; Hyperparameter Search}
      \subsubsection*{}
      \subsubsection*{}
      \subsubsection*{}
      \subsubsection*{}
      \subsubsection*{}
    \newpage

  \section*{\sc Unit 2: Engineering Features (and Friends) by Hand}
    \subsection*{Lecture 2a: Feature Engineering}
      \subsubsection*{Bias Trick}
      \subsubsection*{Black Holes}
      \subsubsection*{Specialty Features, e.g.\ Trig or Threshold}
      \subsubsection*{Interpreting Weights (vs Correlations)}
      \subsubsection*{Feature Selection and Generalization}
    \newpage

    \subsection*{Lecture 2b: Loss Functions, Regularization, Margins} % Logistic, Hinge, Perceptron, Square; acc bound; L-inf, L2, L1; prob interp
      \subsubsection*{Humble Models, Acc Bound}
      \subsubsection*{Regularization: Why and How}
      \subsubsection*{Probabilistic Interpretation, Likelihood, Gradient Descent}
      \subsubsection*{Margin Maximization and Supports}
      \subsubsection*{Feature Selection and Optimization (Convexity)}
    \newpage

    \subsection*{Lecture 2c: Data-based Featurization} % Kernels, PCA, Quantiles & Trees
      \subsubsection*{Quantiles (Binning), Stumps, Trees}
      \subsubsection*{Linear Dimension Reduction: PCA}
      \subsubsection*{Matrix Factorization Perspective, ICA}
      \subsubsection*{Nonlinear Dimension Reduction: Similarity-to-Landmark Embeddings.  Kernels}
      \subsubsection*{Designing Similarity Functions.  Consideration for Text, Images.}
    \newpage

    \subsection*{Lecture 2d: Reducing to Linear Learning} % Softmax, Regression, Sequences, etc
      \subsubsection*{Specialized Readouts}
      \subsubsection*{Soft Classes, Multiple Classes}
      \subsubsection*{Regression}
      \subsubsection*{Factoring Big Problems.  E.g.\ Sequences}
      \subsubsection*{Shared Features for Multiple Outputs.  Examples of Overall Architectures}
    \newpage

    \subsection*{Coding Lecture 2: SVM with RBF Features from Scratch, for Image Classification}
      \subsubsection*{Plan, Pictures, Meeting Data}
      \subsubsection*{Forward Model}
      \subsubsection*{Gradient Step}
      \subsubsection*{Training and Model Selection}
      \subsubsection*{Testing, Interpreting Weights}
    \newpage

    \subsection*{Homework 2a: Feature Geometry, Geometry of Margins and Regression}
      \subsubsection*{Matching Features (and Losses/Architectures) and Domain Knowledge, Practice}
      \subsubsection*{Recognizing Decision Boundaries}
      \subsubsection*{Probabilistic Interpretation of Losses, Regularizers}
      \subsubsection*{Support Vector Bound.  Perceptron Update.  Perceptron worse generalization.}
      \subsubsection*{Implicit Regularization in Least Squares Regression by Gradient Descent}
    \newpage

    \subsection*{Homework 2b: Kernel Trick}
      \subsubsection*{Data-Dependent Features.  Similarity, Inner Products, Generalization.}
      \subsubsection*{Why We Want Positive Definite Kernel}
      \subsubsection*{Kernelized Update}
      \subsubsection*{Recognizing Decision Boundaries} % ??
      \subsubsection*{Wide Random Features Kernel}
    \newpage

    \subsection*{Project 2a: Overfitting to Features on Kaggle-Style Challenge} % Selection/Hyperparam/Unsupervised Featurization
      \subsubsection*{Adapt the SVM to Kaggle Task; Weights and Overfitting}
      \subsubsection*{Success Measures.  Accuracy vs Loss.  ROC Curve, Ranking.  Class-imbalance.}
      \subsubsection*{Feature Selection and Overfitting}
      \subsubsection*{Quantilization and Overfitting}
      \subsubsection*{DataSnooping and Overfitting} % difficult but important lesson to design!
    \newpage

    \subsection*{Project 2b: 10-Way Image Softmax Classification.  Calibration Prediction Challenge}
      \subsubsection*{Softmax, Prob Calibration}
      \subsubsection*{Basic Features.  Add your twist!}
      \subsubsection*{Badly Set Hyperparams.  Add your insight!}
      \subsubsection*{Assessing Confidence: Generalization Bounds, Margin-Based Cross-Validation}
      \subsubsection*{Prediction Challenge}
    \newpage

  \section*{\sc Unit 3: Learning Features from Data}
    \subsection*{Lecture 3a: Shallow Learning} % Optimization tricks
      \subsubsection*{Architecture, Intuiting Decision Boundaries (Universality)}
      \subsubsection*{Gradient Descent (Backprop)}
      \subsubsection*{An Example Learning Trajectory: Vertical vs Horizontal Learning}
      \subsubsection*{MLP as Butter: add to everything.  Regularization.  Examples.}
      \subsubsection*{BIC, Johann-Lindenstrauss, Wide Random Features, Generalization.}
    \newpage

    \subsection*{Lecture 3b: Deep Learning and Ideas in Architecture} % Differentiable-Blah Flexibility in architecture.  Mention RNNs
      \subsubsection*{``To Build It, Use It''}
      \subsubsection*{Feature Hierarchy.  Learned ``subroutines''}
      \subsubsection*{Ideas in Optimization: Backprop, Weight Initialization, Batch Normalization, ADAM, etc}
      \subsubsection*{Getting Creative.  Skip Connections etc}
      \subsubsection*{Curvature-Noise Interactions and Generalization}
    \newpage

    \subsection*{Lecture 3c: Symmetry and Locality: Convolution}
      \subsubsection*{CNN forward encodes basic image priors.  Pooling.}
      \subsubsection*{Backprop formulas}
      \subsubsection*{What can K Layers of a CNN Represent?}
      \subsubsection*{How GPUs Work}
      \subsubsection*{Pottery Image.  Symmetries and Representations.  Sports.  Melodies.}
    \newpage

    \subsection*{Lecture 3d: Symmetry and Locality: Attention}
      \subsubsection*{Attention forward encodes basic sequence/table priors (sparse).}
      \subsubsection*{Bells and Whistles: Multiple Heads, Positional Encoding}
      \subsubsection*{What is a Transformer?}
      \subsubsection*{What can K Transformer Layers Represent?}
      \subsubsection*{Differentiable Computers.  Stacks.  Graphs.}
    \newpage

    \subsection*{Coding Lecture 3: A Deep Image Classifier from Scratch}
      \subsubsection*{Plan of what we need to write, picture of architecture}
      \subsubsection*{Forward Model (Including conv layers)}
      \subsubsection*{Backward Model (Including conv layers)}
      \subsubsection*{Quick Checks for Silly Mistakes}
      \subsubsection*{Training and Interpretation of Weights}
    \newpage

    \subsection*{Homework 3a: Decision Boundaries, Backpropagation}
      \subsubsection*{Backprop for Simple, Fanciful Architecture}
      \subsubsection*{Vanilla Neural Network Architecture}
      \subsubsection*{VNN Backprop Formulas and Intuition}
      \subsubsection*{Recognizing Decision Boundaries for Wide vs Deep}
      \subsubsection*{Initialization of Weights, Match Comments to Lines in Training Code}
    \newpage

    \subsection*{Homework 3b: Zoo of Architectural Ideas} % Light Load.  Only few questions, all conceptual.
      \subsubsection*{Word Embeddings}
      \subsubsection*{Siamese, Metric, LeCun Representation Learners}
      \subsubsection*{RNNs, LSTMs}
      \subsubsection*{Differentiable X (Renderer etc)}
      \subsubsection*{Learned Losses / Ecosystems (Wass GAN, etc)}
    \newpage

    \subsection*{Project 3a: An Image Sharpener (or maybe next-frame-in-video)?} % Perhaps Image Sharpener? % on both Digits and Face data?
      \subsubsection*{U-Net Architecture (Pix2Pix but no GAN?)}
      \subsubsection*{Forward Model}
      \subsubsection*{Backward Model}
      \subsubsection*{Training}
      \subsubsection*{A word on Laplacian Pyramid and Diffusion Models.  Adapt to Image Generator and see bad.}
    \newpage

    \subsection*{Project 3b: Text Generation Through Classification}
      \subsubsection*{Transformer Architecture, Meet the Data}
      \subsubsection*{Run Training Code}
      \subsubsection*{Interpreting Learned Weights}
      \subsubsection*{Playing with Model on Out-of-Distr Data}
      \subsubsection*{Text Compression Challenge}
    \newpage

  \section*{\sc Unit 4: Modeling Structured Uncertainties}
    \subsection*{Lecture 4a: Square Loss Muddies, Probabilistic Models.  Examples: 3State Traffic, HMM, GMM}
      \subsubsection*{Structured Uncertainties in the World.  Rapid Holmesian
      Inferences.  Square Loss Muddies.  Recall prob interp of e.g. least
      squares regfression.  Unsup Learning.}
      \subsubsection*{Latents, Marginal Likelihood for Latents, Weights as Latents.}
      \subsubsection*{Example: 3State Traffic.  Challenge of Normalization.}
      \subsubsection*{Example: HMM.  0s and SoftLogic and backflow}
      \subsubsection*{Example: GMM.  Metapriors, hierarchy, Transfer}
    \newpage

    \subsection*{Lecture 4b: Inference via Variation} % MH, EM,
      \subsubsection*{EM overview, 3State Traffic example}
      \subsubsection*{ELBO bound and pingpong KL geometry}
      \subsubsection*{EM: HMM example (more detail in coding example)}
      \subsubsection*{EM: GMM example (more detail in pset)}
      \subsubsection*{Variational Parts as Neural Net}
    \newpage

    \subsection*{Lecture 4c: Inference via Sampling} % MH, EM,
      \subsubsection*{Challenge.  MH Algorithm.}
      \subsubsection*{Visualizing MH.  Proposals Matter}
      \subsubsection*{EM: 3State Traffic example}
      \subsubsection*{EM: HMM example}
      \subsubsection*{EM: GMM example}
    \newpage

    \subsection*{Lecture 4d: Variational Autoencoders (or Encoders)} % so connects back to Unit 3
      \subsubsection*{Architecture} % comparison to matrix factorization, etc
      \subsubsection*{VAEs and ELBO}
      \subsubsection*{Interpreting Update Intuitively}
      \subsubsection*{Output side Noise Model (e.g. square loss)}
      \subsubsection*{Conditional VAEs}
    \newpage

    \subsection*{Coding Lecture 4: Expectation-Maximization for Sequences}
      \subsubsection*{HMM architecture.  Remark on RNNa}
      \subsubsection*{E-step: dynamic programming}
      \subsubsection*{M-step with regularization}
      \subsubsection*{End-to-end reading and mimicry of cookbook text}
      \subsubsection*{Investigation of State Meanings}
    \newpage

    \subsection*{Homework 4a: Gaussian Mixtures for Clustering} % k-means as limit
      \subsubsection*{forward model}
      \subsubsection*{E step and M step}
      \subsubsection*{k-means as limit}
      \subsubsection*{small-var regularization from prior}
      \subsubsection*{convergence near minimum}
    \newpage

    \subsection*{Homework 4b: Bayes Nets.  NNs as Amortized Inference}
      \subsubsection*{Marginalization over Latents}
      \subsubsection*{Weights as Latents in Linear Classification (bowtie vs pipe)}
      \subsubsection*{BIC and structural penalty}
      \subsubsection*{Causal modeling}
      \subsubsection*{Very Basic Occlusion Inference Object Net}
    \newpage

    \subsection*{Project 4a: Predicting Political Polls}
      \subsubsection*{Meeting the Polling Data}
      \subsubsection*{Forward Model}
      \subsubsection*{Inference through Sampling}
      \subsubsection*{Wrestling with Training Difficulties (burnin, etc)}
      \subsubsection*{Interpreting Latents}
    \newpage


    \subsection*{Project 4b: A Deep Image Generator (VAE)}
      \subsubsection*{Meeting the Face Data.  How to assess success?} % maybe met face data in previous unit?
      \subsubsection*{Forward Model including reparam trick}
      \subsubsection*{Backward model and intuitive interpretation}
      \subsubsection*{Training and Testing}
      \subsubsection*{Interpreting Latents.  Sources of Prejudice and Capriciousness}
    \newpage

  \section*{\sc Unit 5: Learning while Acting (and from Rewards)}
    \subsection*{Lecture 5a: Rewards and States.  Explore-Exploit Challenge.}
      \subsubsection*{Learning from Rewards: overview and goal to reduce to supervision}
      \subsubsection*{Conditional Bandits are kin to Supervised Learning}
      \subsubsection*{Explore-Exploit Challenge}
      \subsubsection*{Optimism in Face of Uncertainty}
      \subsubsection*{Introduction of State Concept.  Total Utility, Episodes,  Online blurs Train and Test}
    \newpage

    \subsection*{Lecture 5b: Qualitative Solutions to MDP.  Dynamic Programming, Idea of Bootstrap}
      \subsubsection*{MDP framework again.  State Correlations (Credit-Assignment) Challenge}
      \subsubsection*{Dynamic Programming}
      \subsubsection*{Bootstrap: Online dynamic programming (SARSA)}
      \subsubsection*{Example Policy in Gridworld}
      \subsubsection*{Function Approximation (or Partial Observability)}
    \newpage

    \subsection*{Lecture 5c: Reduce to Supervision using Q-Learning.}
      \subsubsection*{Q-Learning vs Naive SARSA}
      \subsubsection*{Q-Learning plus Exploration Policy}
      \subsubsection*{Featurization of Huge State (Deep Q-Learning)}
      \subsubsection*{LSTM for non-Markov State}
      \subsubsection*{Deadly Triad, Q-Delusion}
    \newpage

    \subsection*{Lecture 5d: Non-technical Discussion of: Curiosity, Language, Instruction, World-Models}
      \subsubsection*{Curiosity Bonuses and Self-Curriculum} % mention Adversarial self-play?
      \subsubsection*{Curricula in RL and Elsewhere}
      \subsubsection*{Symbols, Contextual Learning in GPT, Instructions guide Symbol Search in World-Models}
      \subsubsection*{Evolution of Programming.  Self-Coding Computers}
      \subsubsection*{Speculations on Future of Learning}
    \newpage

    \subsection*{Coding Lecture 5: Q-Learning for a Gridworld, from Scratch}
      \subsubsection*{World Simulator}
      \subsubsection*{Q-Table (with flexibility to be function approx)}
      \subsubsection*{Q-Learning Update}
      \subsubsection*{Training and Reading out Policy}
      \subsubsection*{Visualizing Learning curves as add challenges}
    \newpage

    \subsection*{Homework 5a: Bandits, Explore-Exploit, Conditional Bandits}
      \subsubsection*{Epsilon Greedy for Known Timescale}
      \subsubsection*{UCB Bound vs Epsilon Greedy at Various Timescales}
      \subsubsection*{Function Approximation in Conditional Bandits}
      \subsubsection*{Movie Recommendation Update (Linear Embeddings)}
      \subsubsection*{A Very Basic Adversarial BoardGame Player}
    \newpage

    \subsection*{Homework 5b: Q-Learning, on-vs-off policy, Horizons}
      \subsubsection*{Practice Modeling Situations by MDPs.  Get Rich Enough State Space for Markov Property!}
      \subsubsection*{Manually Solve Various MDPs}
      \subsubsection*{Computations of Q-Learning Convergence}
      \subsubsection*{As Human, Explore a Maze.  Short Essay on Exploration Strategy}
      \subsubsection*{Baud Rates for Learning Signals Comparison}
    \newpage

    \subsection*{Project 5a: Robotic Arm}
      \subsubsection*{Robot Physics Overview.  Collisions?}
      \subsubsection*{Clasping Reward Function (Helper to Smooth)}
      \subsubsection*{Set up Q-learning for parameterized task (what are inputs to Q network etc?)} % mention policy gradient?
      \subsubsection*{Training}
      \subsubsection*{Visualize and Describe Motions}
    \newpage

    \subsection*{Project 5a: Simple Atari-Style Games} % (Fully Observed?  Partially Observed?)
      \subsubsection*{Game Family Overview}
      \subsubsection*{Function Approximation}
      \subsubsection*{Experience Replay}
      \subsubsection*{Creative Part: Design Featurization and Exploration Policy!}
      \subsubsection*{Action Challenge: Learn an Unknown Game in Same Family!}
    \newpage

  \section*{\sc Unit 6: Farewell}
    \subsection*{Bonus Lecture 6a: What We Learned}
      \subsubsection*{}
      \subsubsection*{}
      \subsubsection*{}
      \subsubsection*{}
      \subsubsection*{}


    \subsection*{Bonus Lecture 6b: What to Learn Next}
      \subsubsection*{}
      \subsubsection*{}
      \subsubsection*{}
      \subsubsection*{}
      \subsubsection*{}
    \newpage


\part{Development Plan}

\end{document}
