% author:   sam tenka
% change:   2023-05-25
% create:   2022-05-11

\documentclass[11pt, justified]{tufte-book}
\geometry{
  left           = 0.90in, % left margin
  textwidth      = 4.95in, % main text block
  marginparsep   = 0.15in, % gutter between main text block and margin notes
  marginparwidth = 2.30in, % width of margin notes
                 % 0.20in  % width from margin to edge
}
% author:   samtenka
% change:   2023-03-28
% create:   2022-05-11

\newcommand{\picturedw}[2]{\includegraphics[width=#1]{#2}}
\newcommand{\picturew}[2]{\includegraphics[width=#1]{figures/#2}}
\newcommand{\pictureh}[2]{\includegraphics[height=#1]{figures/#2}}

%==============================================================================
%====  0.  DOCUMENT SETTINGS  ================================================
%==============================================================================

%~~~~~~~~~~~~~~~~~~~~~~~~~~~~~~~~~~~~~~~~~~~~~~~~~~~~~~~~~~~~~~~~~~~~~~~~~~~~~~
%~~~~~~~~~~~~~  0.0. About this Exposition  ~~~~~~~~~~~~~~~~~~~~~~~~~~~~~~~~~~~

%---------------------  0.0.1. math packages  ---------------------------------
\newcommand\hmmax{0} % to allow for more fonts
\newcommand\bmmax{0} % to allow for more fonts
\usepackage{amsmath, amssymb, amsthm, mathtools}
\usepackage{bm}
\usepackage{euler}

\usepackage{array}   % for \newcolumntype macro
\newcolumntype{L}{>{$}l<{$}} % math-mode version of "l" column type
\newcolumntype{C}{>{$}c<{$}} % math-mode version of "c" column type
\newcolumntype{R}{>{$}r<{$}} % math-mode version of "r" column type

\newcolumntype{S}{ >{\centering\arraybackslash} m{3cm} } % vertically and horizontally centered

%---------------------  0.0.2. graphics packages  -----------------------------
\usepackage{graphicx, xcolor}
\usepackage{float, capt-of}
\usepackage{soul}

%---------------------  0.0.3. packages for fancy text  -----------------------
\usepackage{enumitem}\setlist{nosep}
\usepackage{listings}
\usepackage{xstring}
\usepackage{fontawesome5}

%---------------------  0.043. colors  ----------------------------------------

% NOTE: we want to cater to colorblind readers

% (LIGHT, MEDIUM, DARK) x (BLUE, ORANGE)
\definecolor{msky}{rgb}{0.62, 0.82, 0.94} \newcommand{\sky}{\color{msky}}
\definecolor{mpch}{rgb}{0.98, 0.86, 0.62} \newcommand{\pch}{\color{mpch}}

\definecolor{mblu}{rgb}{0.05, 0.55, 0.85} \newcommand{\blu}{\color{mblu}}
\definecolor{mrng}{rgb}{0.95, 0.65, 0.05} \newcommand{\rng}{\color{mrng}}

\definecolor{msea}{rgb}{0.02, 0.22, 0.34} \newcommand{\sea}{\color{msea}}
\definecolor{mbro}{rgb}{0.38, 0.26, 0.02} \newcommand{\bro}{\color{mbro}}

% SHADES:
\definecolor{mgre}{rgb}{0.55, 0.55, 0.55} \newcommand{\gre}{\color{mgre}}
\definecolor{mdgre}{rgb}{0.35, 0.35, 0.35} \newcommand{\dgre}{\color{mdgre}}

% UNFRIENDLY:
\definecolor{mred}{rgb}{1.00, 0.00, 0.00} \newcommand{\red}{\color{mred}}

%~~~~~~~~~~~~~~~~~~~~~~~~~~~~~~~~~~~~~~~~~~~~~~~~~~~~~~~~~~~~~~~~~~~~~~~~~~~~~~
%~~~~~~~~~~~~~  0.1. Headers and References  ~~~~~~~~~~~~~~~~~~~~~~~~~~~~~~~~~~

%---------------------  0.1.0. intra-document references  ---------------------
\newcommand{\offour}[1]{
    {\tiny \raisebox{0.04cm}{\scalebox{0.9}{$\substack{
        \IfSubStr{#1}{0}{{\blacksquare}}{\square}
        \IfSubStr{#1}{1}{{\blacksquare}}{\square} \\ 
        \IfSubStr{#1}{2}{{\blacksquare}}{\square}
        \IfSubStr{#1}{3}{{\blacksquare}}{\square}
    }$}}}%
}

\newcommand{\offourline}[1]{
    {\tiny \raisebox{0.04cm}{\scalebox{0.9}{$\substack{
        \IfSubStr{#1}{0}{{\blacksquare}}{\square}
        \IfSubStr{#1}{1}{{\blacksquare}}{\square}
        \IfSubStr{#1}{2}{{\blacksquare}}{\square}
        \IfSubStr{#1}{3}{{\blacksquare}}{\square}
    }$}}}%
}
\newcommand{\notesam}[1]{{\blu \textsf{#1}}}
\newcommand{\attn}[1]{{\blu \textsf{#1}}}
\newcommand{\attnsam}[1]{{\red \textsf{#1}}}
%\newcommand{\attnsam}[1]{}%{\red \textsf{#1}}}

\newcommand{\blarr}{\hspace{-0.15cm}${\blu \leftarrow}\,$}
\newcommand{\bcirc}{${\blu ^\circ}$}
\newcommand{\bovinenote}[1]{\bcirc\marginnote{\blarr #1}}

%---------------------  0.1.1. table of contents helpers  ---------------------
\newcommand{\phdot}{\phantom{.}}

%---------------------  0.1.2. section headers  -------------------------------
\newcommand{\samtitle} [1]{
  \par\noindent{\Huge \sf \blu #1}
  \vspace{0.4cm}
}

\newcommand{\samquote} [2]{
    \marginnote[-0.4cm]{\begin{flushright}
    %\scriptsize
        \gre {\it #1} \\ --- #2
    \end{flushright}}
}

\newcommand{\sampart} [1]{
  \vspace{0.5cm}
  \par\noindent{\LARGE \sf \blu #1}
  \vspace{0.1cm}\par
}

\newcommand{\samsection}[1]{
  \vspace{0.3cm}
  \par\noindent{\Large \sf \blu #1}
  \vspace{0.1cm}\par
}

\newcommand{\sampassage}[1]{
   \vspace{0.1cm}
   \par\noindent{\hspace{-2cm}\normalsize \sc \gre #1} ---
}

%---------------------  0.1.3. clear the bibliography's header  ---------------
\usepackage{etoolbox}
\patchcmd{\thebibliography}{\section*{\refname}}{}{}{}

%~~~~~~~~~~~~~~~~~~~~~~~~~~~~~~~~~~~~~~~~~~~~~~~~~~~~~~~~~~~~~~~~~~~~~~~~~~~~~~
%~~~~~~~~~~~~~  0.2. Math Symbols and Blocks  ~~~~~~~~~~~~~~~~~~~~~~~~~~~~~~~~~

%---------------------  0.2.0. general math operators  ------------------------
\newcommand{\scirc}{\mathrel{\mathsmaller{\mathsmaller{\mathsmaller{\circ}}}}}
\newcommand{\cmop}[2]{{(#1\!\to\!#2)}}
\newcommand{\pr}{^\prime}
\newcommand{\prpr}{^{\prime\prime}}

\newcommand{\wrap}[1]{\left(#1\right)}

%---------------------  0.2.1. probability symbols  ---------------------------
\newcommand{\KL}{\text{KL}}
\newcommand{\EN}{\text{H}}
\newcommand{\note}[1]{{\blu \textsf{#1}}}

%---------------------  0.2.2. losses averaged in various ways  ---------------
\newcommand{\Ein}  {\text{trn}_{\sS}}
\newcommand{\Einb} {\text{trn}_{\check\sS}}
\newcommand{\Einc} {\text{trn}_{\sS\sqcup \check\sS}}
\newcommand{\Egap} {\text{gap}_{\sS}}
\newcommand{\Eout} {\text{tst}}

%---------------------  0.2.3. double-struck and caligraphic upper letters  ---
\newcommand{\Aa}{\mathbb{A}}\newcommand{\aA}{\mathcal{A}}
\newcommand{\Bb}{\mathbb{B}}\newcommand{\bB}{\mathcal{B}}
\newcommand{\Cc}{\mathbb{C}}\newcommand{\cC}{\mathcal{C}}
\newcommand{\Dd}{\mathbb{D}}\newcommand{\dD}{\mathcal{D}}
\newcommand{\Ee}{\mathbb{E}}\newcommand{\eE}{\mathcal{E}}
\newcommand{\Ff}{\mathbb{F}}\newcommand{\fF}{\mathcal{F}}
\newcommand{\Gg}{\mathbb{G}}\newcommand{\gG}{\mathcal{G}}
\newcommand{\Hh}{\mathbb{H}}\newcommand{\hH}{\mathcal{H}}
\newcommand{\Ii}{\mathbb{I}}\newcommand{\iI}{\mathcal{I}}
\newcommand{\Jj}{\mathbb{J}}\newcommand{\jJ}{\mathcal{J}}
\newcommand{\Kk}{\mathbb{K}}\newcommand{\kK}{\mathcal{K}}
\newcommand{\Ll}{\mathbb{L}}\newcommand{\lL}{\mathcal{L}}
\newcommand{\Mm}{\mathbb{M}}\newcommand{\mM}{\mathcal{M}}
\newcommand{\Nn}{\mathbb{N}}\newcommand{\nN}{\mathcal{N}}
\newcommand{\Oo}{\mathbb{O}}\newcommand{\oO}{\mathcal{O}}
\newcommand{\Pp}{\mathbb{P}}\newcommand{\pP}{\mathcal{P}}
\newcommand{\Qq}{\mathbb{Q}}\newcommand{\qQ}{\mathcal{Q}}
\newcommand{\Rr}{\mathbb{R}}\newcommand{\rR}{\mathcal{R}}
\newcommand{\Ss}{\mathbb{S}}\newcommand{\sS}{\mathcal{S}}
\newcommand{\Tt}{\mathbb{T}}\newcommand{\tT}{\mathcal{T}}
\newcommand{\Uu}{\mathbb{U}}\newcommand{\uU}{\mathcal{U}}
\newcommand{\Vv}{\mathbb{V}}\newcommand{\vV}{\mathcal{V}}
\newcommand{\Ww}{\mathbb{W}}\newcommand{\wW}{\mathcal{W}}
\newcommand{\Xx}{\mathbb{X}}\newcommand{\xX}{\mathcal{X}}
\newcommand{\Yy}{\mathbb{Y}}\newcommand{\yY}{\mathcal{Y}}
\newcommand{\Zz}{\mathbb{Z}}\newcommand{\zZ}{\mathcal{Z}}

%---------------------  0.2.4. sans serif and frak lower letters  -------------
\newcommand{\sfa}{\mathsf{a}}\newcommand{\fra}{\mathfrak{a}}
\newcommand{\sfb}{\mathsf{b}}\newcommand{\frb}{\mathfrak{b}}
\newcommand{\sfc}{\mathsf{c}}\newcommand{\frc}{\mathfrak{c}}
\newcommand{\sfd}{\mathsf{d}}\newcommand{\frd}{\mathfrak{d}}
\newcommand{\sfe}{\mathsf{e}}\newcommand{\fre}{\mathfrak{e}}
\newcommand{\sff}{\mathsf{f}}\newcommand{\frf}{\mathfrak{f}}
\newcommand{\sfg}{\mathsf{g}}\newcommand{\frg}{\mathfrak{g}}
\newcommand{\sfh}{\mathsf{h}}\newcommand{\frh}{\mathfrak{h}}
\newcommand{\sfi}{\mathsf{i}}\newcommand{\fri}{\mathfrak{i}}
\newcommand{\sfj}{\mathsf{j}}\newcommand{\frj}{\mathfrak{j}}
\newcommand{\sfk}{\mathsf{k}}\newcommand{\frk}{\mathfrak{k}}
\newcommand{\sfl}{\mathsf{l}}\newcommand{\frl}{\mathfrak{l}}
\newcommand{\sfm}{\mathsf{m}}\newcommand{\frm}{\mathfrak{m}}
\newcommand{\sfn}{\mathsf{n}}\newcommand{\frn}{\mathfrak{n}}
\newcommand{\sfo}{\mathsf{o}}\newcommand{\fro}{\mathfrak{o}}
\newcommand{\sfp}{\mathsf{p}}\newcommand{\frp}{\mathfrak{p}}
\newcommand{\sfq}{\mathsf{q}}\newcommand{\frq}{\mathfrak{q}}
\newcommand{\sfr}{\mathsf{r}}\newcommand{\frr}{\mathfrak{r}}
\newcommand{\sfs}{\mathsf{s}}\newcommand{\frs}{\mathfrak{s}}
\newcommand{\sft}{\mathsf{t}}\newcommand{\frt}{\mathfrak{t}}
\newcommand{\sfu}{\mathsf{u}}\newcommand{\fru}{\mathfrak{u}}
\newcommand{\sfv}{\mathsf{v}}\newcommand{\frv}{\mathfrak{v}}
\newcommand{\sfw}{\mathsf{w}}\newcommand{\frw}{\mathfrak{w}}
\newcommand{\sfx}{\mathsf{x}}\newcommand{\frx}{\mathfrak{x}}
\newcommand{\sfy}{\mathsf{y}}\newcommand{\fry}{\mathfrak{y}}
\newcommand{\sfz}{\mathsf{z}}\newcommand{\frz}{\mathfrak{z}}

%---------------------  0.2.5. math environments  -----------------------------
\newtheorem*{qst}{Question}
\newtheorem*{thm}{Theorem}
\newtheorem*{lem}{Lemma}
% ...
\theoremstyle{definition}
\newtheorem*{dfn}{Definition}

\newcommand{\exercise}[1]{%
  \par\noindent%
  \attn{Food For Thought:} #1%
}
\newcommand{\noparexercise}[1]{%
  \attn{Food For Thought:} #1%
}
\newcommand{\objectives}[1]{%
  \marginnote[-0.2cm]{%
    By the end of this section, you'll be able to
    \begin{itemize}#1\end{itemize}
  }
}



\usepackage{tikz-cd}

\begin{document}\samtitle{basics of machine learning}

\sampassage{contributors}
  \texttt{karene},
\texttt{laurent},
\texttt{sam},

\attnsam{append your name to the list above by editing `CONTRIBUTORS.tex`}



\sampassage{hallo!}
  These optional and still-evolving notes for 6.86x give a big picture view of
  our conceptual landscape.  They might help you study, but you can do all your
  assignments without these notes.  These notes treat many but not all the
  topics of our course; they also treat a few topics not emphasized in class,
  to fill gaps for curious learners.

  I'm happy to answer questions about the notes on the forum.  If you want to
  help improve these notes, ask me; I'll be happy to list you as a contributor!

  We compiled these notes using the following color palette; let us know if
  the colors are hard to distinguish.

  {\Huge\vspace{-1.5cm}\[\substack{%
    {\sky\blacksquare}
    {\blu\blacksquare}
    {\sea\blacksquare}\\
    {\pch\blacksquare}
    {\rng\blacksquare}
    {\bro\blacksquare}}
    \]}

    \marginnote[-6cm]{%
    \textsc{Clickable Table of Contents}\vspace{0.05cm}
    \begin{description}
        \item[\hyperlink{A}{A. prologue}]                                               \phdot  \hfill\pageref{part:A}
        \begin{description}
          \item[\hyperlink{A0}{{\color{gray}0} what is learning?}]
          \item[\hyperlink{A1}{{\color{gray}1} a tiny example: handwritten digits}]
          \item[\hyperlink{A2}{{\color{gray}2} how well did we do?}]
          \item[\hyperlink{A3}{{\color{gray}3} how can we do better?}]
        \end{description}
    \item[\hyperlink{part:Z}{Z. a complete, simple classifier}]                         \phdot  \hfill\pageref{part:Z}
        \begin{description}
          \item[\hyperlink{Z0}{{\color{gray}0} goodness of fit}]
          \item[\hyperlink{Z1}{{\color{gray}1} smarter optimization}]
          \item[\hyperlink{Z2}{{\color{gray}2} two examples: cows and walls}]
          \item[\hyperlink{Z3}{{\color{gray}3} intrinsic plausibility}]
        \end{description}
    \item[\hyperlink{part:B}{B. squeezing more out of linear models}]             \phdot  \hfill\pageref{part:B}
        \begin{description}
          \item[\hyperlink{B0}{{\color{gray}0} featurization}]
          \item[\hyperlink{B1}{{\color{gray}1} quantifying uncertainty}]
          \item[\hyperlink{B2}{{\color{gray}2} iterative optimization}]
          \item[\hyperlink{B3}{{\color{gray}3} data-dependent features}]
        \end{description}
    \item[\hyperlink{part:C}{C. bend those lines to capture rich patterns}]             \phdot  \hfill\pageref{part:C}
        \begin{description}
          \item[\hyperlink{C0}{{\color{gray}0} featurization}]
          \item[\hyperlink{C1}{{\color{gray}1} learned featurizations}]
          \item[\hyperlink{C2}{{\color{gray}2} locality and symmetry in architecture}]
          \item[\hyperlink{C3}{{\color{gray}3} dependencies in architecture}]
        \end{description}
    \item[\hyperlink{part:D}{D. thicken those lines to quantify uncertainty}]           \phdot  \hfill\pageref{part:D}
        \begin{description}
          \item[\hyperlink{D0}{{\color{gray}0} bayesian models}]
          \item[\hyperlink{D1}{{\color{gray}1} examples of bayesian models}]
          \item[\hyperlink{D2}{{\color{gray}2} inference algorithms for bayesian models}]
          \item[\hyperlink{D3}{{\color{gray}3} combining with deep learning}]
        \end{description}
    \item[\hyperlink{part:E}{E. beyond learning-from-examples}]                         \phdot  \hfill\pageref{part:E}
        \begin{description}
          \item[\hyperlink{E0}{{\color{gray}0} reinforcement}]
          \item[\hyperlink{E1}{{\color{gray}1} state}]
          \item[\hyperlink{E2}{{\color{gray}2} deep q learning}]
          \item[\hyperlink{E3}{{\color{gray}3} learning from instructions}]
        \end{description}
    %
    \item[] \vspace{0.05cm} \hrule \vspace{-0.05cm}
    %
    \item[\hyperlink{part:F}{F. some background}]                                       \phdot  \hfill\pageref{part:F}
        \begin{description}
          \item[\hyperlink{F0}{{\color{gray}0} probability primer}]
          \item[\hyperlink{F1}{{\color{gray}1} linear algebra primer}]
          \item[\hyperlink{F2}{{\color{gray}2} derivatives primer}]
          \item[\hyperlink{F3}{{\color{gray}3} programming and numpy and pytorch primer}]
        \end{description}
    %\item[\hyperlink{part:G}{G. rambles}]                                               \phdot  \hfill\pageref{part:G}
    %    \begin{description}
    %      \item[\hyperlink{G0}{{\color{gray}0} }]
    %      \item[\hyperlink{G1}{{\color{gray}1} }]
    %      \item[\hyperlink{G2}{{\color{gray}2} }]
    %      \item[\hyperlink{G3}{{\color{gray}3} }]
    %    \end{description}
    \end{description}
  }

  \sampassage{navigating these notes}
    We divide these notes into bite-size passages; each begins with a gray
    heading (e.g.\ `navigating these notes' to the left).  All passages are
    optional in the sense that you'll be able to complete all assignments
    without ever studying these notes.  Yet, aside from a few bonus passages
    marked as \textsf{\blu VERY OPTIONAL}, most passages offer complementary
    discussion of our course topics that may help you understand, appreciate,
    and internalize the lectures.  After all, only with two lines of sight can
    we see depth.

    These notes highlight probability theory and model architecture more than
    the lectures; clustering and low-rank approximation, less.  We can use
    different projections to tear and squash a globe into a flat map of earth.
    Different views reveal different aspects of geography.  In order to ``keep
    australia from being split into two parts'' and to ``display the south
    pole'', I used a different map projection than the lectures do.  So we
    traverse topics in the notes in a different order than we do in lecture.
    Still, we cover essentially the same overall ground.

  \newcommand{\SSS}[2]{\newpage\hypertarget{#1}{}\sampart{#2}\phantomsection\label{part:#1}}
  \newcommand{\sss}[3]{\hypertarget{#1}{}\samsection{#3}\input{tex-source/#2}}

  \newpage
  \SSS{0}{prologue}
    \sss{0a}{body.0.0.what-is-learning}{what is learning?}
    \sss{0b}{body.0.1.our-first-learning-algorithm}{a tiny example: classifying handwritten digits}
    \sss{0c}{body.0.2.how-well-did-we-do}{how well did we do?  analyzing our error}
    \sss{0d}{body.0.3.how-can-we-do-better}{how can we do better?  (survey of rest of notes)}

  \newpage
  %a complete, simple classifier (unit 1)}
  \SSS{1}{Learning from Examples}
    \sss{1a}{body.1.0.likelihoods}{goodness of fit}
    \sss{1b}{body.1.1.gradients}{smarter optimization}
    \sss{1c}{body.1.2.priors}{intrinsic plausibility}
    \sss{1d}{body.1.3.model-selection}{model selection}

  \newpage
  %squeezing more out of linear models (unit 2)}
  \SSS{2}{Engineering Features (and Friends) by Hand}
    \sss{2a}{body.2.0.featurization}{featurization}
    \sss{2b}{body.2.1.quantify-uncertainty}{richer outputs: quantifying uncertainty}
    \sss{2c}{body.2.2.iterative-optimization}{iterative optimization}
    \sss{2d}{body.2.3.data-dependent-features}{data dependent features}

  \newpage
  %bend those lines to capture rich patterns (unit 3)}
  \SSS{3}{Learning Features from Data}
    \sss{3a}{body.3.0.shallow-learning}{shallow learning}
    \sss{3b}{body.3.1.deep-architecture}{deep learning: ideas in architecture}
    \sss{3c}{body.3.2.convolution}{symmetry and locality: convolution}
    \sss{3d}{body.3.3.attention}{symmetry and locality: attention}

  \newpage
  %thicken those lines to quantify uncertainty (unit 4)}
  \SSS{4}{Modeling Structured Uncertainties}
    \sss{4a}{body.4.0.probabilistic-models}{probabilistic models}
    \sss{4b}{body.4.1.expectation-maximization}{inference by variation: EM}
    \sss{4c}{body.4.2.metropolis-hastings}{inference by sampling: MH}
    \sss{4d}{body.4.3.deep-generators}{deep generative models}

  \newpage
  %beyond learning-from-examples (unit 5)}
  \SSS{5}{Learning while Acting (and from Rewards)}
    \sss{5a}{body.5.0.reinforcement}{rewards, actions, states. exploration}
    \sss{5b}{body.5.1.mdps}{qualitative solutions to mdp, dynamic programming, bootstrap}
        %state (dependence on prev xs and on actions ) ; RL ; partial observations}
    \sss{5c}{body.5.2.q-learning}{reduce to supervision using q-learning}
    \sss{5d}{body.5.3.beyond}{non-technical discussion of: curiosity, language, instruction, world-models}

% =============================================================================
% ==  _  ======================================================================
% =============================================================================

  \newpage
  \SSS{F}{Prereq Helpers}
    \sss{Fa}{body.F.0.linear-algebra}{high dimensions and linear algebra}
      %-- linear "spaces" and linear "maps"
      %-- dimension, trace, determinant, visually
      %-- vectors, covectors, higher tensors
      %-- dot product, angles, and svd, visually
    \sss{Fb}{body.F.1.probability}{uncertainty and probability}
      %-- fundamentals; frequentism vs bayesianism
      %-- 2 sharp tools: independence and averaging
      %-- log loss and friends
      %-- common families of distributions
    \sss{Fc}{body.F.2.derivatives}{optimization and derivatives}
      %-- derivatives are best linear appro.s: chain rule and linearity
      %-- product rule, visually
      %-- optimization : critical points, lagrange, higher derivatives
      %-- write your own automatic differentiator: a short exercise
    \sss{Fd}{body.F.3.programming}{python, numpy, pytorch}

\end{document}

